\documentclass{article}
\usepackage[utf8]{inputenc}
\usepackage{amsmath}
\usepackage{amsfonts}
\usepackage{graphicx}
\usepackage{multicol}
\usepackage{float}
\usepackage{cite}
\usepackage{url}
\usepackage{listings}
\usepackage{pythonhighlight} 
\begin{document}
	\begin{titlepage}
		\begin{center}
			{\huge\textbf{Instituto Politécnico Nacional}}\\
			\vspace{7mm}
			{\huge\textbf{Escuela Superior de Cómputo}}\\			
			\begin{figure}[h]
				\centering
				\includegraphics[height = 6cm]{logoEscom.png}
			\end{figure}	
			\vspace{1cm}
			{\huge\textbf{Reporte Final}}
			\par\vspace{2cm}
			\large\textbf{Autor: Colín Ramiro Joel}
			\par\vspace{1cm}
			{\large\textbf{Materia: Teoría de la Computación}}
			\par\vspace{1cm}
			{\large\textbf{Grupo: 4CM2}}
			\par\vspace{1cm}
			{\large\textbf{Profesor: Juarez Martínez Genaro}}
			\par\vspace{1cm}
			{\large\textbf{Fecha de entrega: {\huge{05 de Enero 2022}}}}
			\par\vspace{3cm}
		\end{center}
	\end{titlepage}
	\section*{Introducción}
	En esta materia de \textbf{Teoría de la Computación} se revisó principalmente la Teoría de Autómatas. Junto con sus respectivos métodos y características.
	 
	
	\section*{Programas}
	\begin{enumerate}
		\item Programa1. Universo	
			\begin{itemize}
				\item Descripción
				\item Código
			\end{itemize}
		\item Programa2. Protocolo
		\begin{itemize}
			\item Descripción
			\item Código
		\end{itemize}
		\item Programa3. Ajedrez
		\begin{itemize}
			\item Descripción
			\item Código
		\end{itemize}
		\item Programa4. Buscador de Palabras
		\begin{itemize}
			\item Descripción
			\item Código
		\end{itemize}
		\item Programa5. Expresión Regular
		\begin{itemize}
			\item Descripción
			\item Código
		\end{itemize}
		\item Programa6. Automáta de pila
		\begin{itemize}
			\item Descripción
			\item Código
		\end{itemize}
		\item Programa7. Palindromes
		\begin{itemize}
			\item Descripción
			\item Código
		\end{itemize}
		\item Programa8. Gramática NO ambigua
		\begin{itemize}
			\item Descripción
			\item Código
		\end{itemize}
		\item Programa9. Máquina de Turing		
		\begin{itemize}
			\item Descripción
			\item Código
		\end{itemize}
	\end{enumerate}
	
	\section*{Conclusiones}
	
	\section*{Referencias}
	
	
\end{document}